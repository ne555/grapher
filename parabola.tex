\documentclass[a4paper]{article}

\usepackage[spanish]{babel}
\usepackage[utf8]{inputenc}

\title{Bresenham for paraboles}
\author{Bedrij Aldo}
\date{}

\begin{document}
	\maketitle

	\begin{abstract}
		Derivating equations for the rasterization of a parabola
		using only integers
	\end{abstract}

	Without loose of generality, a quadratic curve can be represented as 
	\[ y - a x^2 = 0 \]
	That is, a parabola with vertex in $(0;0)$

	We got a point $P = (\Delta x; \Delta y)$ for which the curve must go trough.
	Let's assume that its coordinates are integers.

	We are going to divide the procedure acording to wich variable has de higher delta

	Considering $a>0$ and the part with $ 0 \leq y' $, the other gets for
	reflextion

	\section{ $ 0 \leq y' \leq 1 $ } 
		In this case every pixel will increase its $x$ coordinate

		At each step we test if the mid-point of the pixel lies below or above the curve.
		If it lies below, we paint the $N$ pixel.

		Using the discriminant 
		\[ D = y - a x^2 \]

		So if $D>0$ the midpoint is above the curve.

		So we test the midpoint
		\[ D = (y_j+0.5) - \frac{\Delta y}{\Delta x^2} (x_j+1)^2\]


		In order to avoid the rationals number we multiply all by $2 \Delta x^2$
		\[ D = -2 \Delta y (x_j+1)^2 + \Delta x^2 (2 y_j+1) \]

		As we start in the vertex $(0;0)$ we've got
		\[ D_0 = -2 \Delta y + \Delta x^2 \]

		We could test this for each x that we encounter, but taking into account the increment

		\[ \Delta D = \left\{
			\begin{array}{ll}
				\mbox{South: } & -2\Delta y (2x_j + 3) \\
				\mbox{North: } & -2\Delta y (2x_j + 3) + 2 \Delta x^2 \\
			\end{array}
			\right.
		\]

		\subsection{Incrementing the incrementals}

	\section{ $1<y'$ }
		In this case every pixel will increase its $y$ coordinate.
		We need to analize if we paint the $E$ or $W$ pixel.

		The discriminant
		\[D = (y_j+1) - \frac{\Delta y}{\Delta x^2} (x+0.5)^2 \]

		In order to avoid the rationals number we multiply all by $4 \Delta x^2$
		\[ D = 4\Delta x^2 (y_j+1) -\Delta y (2x_j+1)^2 \]

		The increments
		\[ \Delta D = \left\{
			\begin{array}{ll}
				\mbox{East: } & 4 \Delta x^2 - 8 \Delta y (x_j + 1) \\
				\mbox{West: } & 4 \Delta x^2
			\end{array}
			\right.
		\]

		\subsection{Incrementing the incrementals}
			We only need to actualize in the $EE$ case
			\[EE = EE - 8\Delta y\]

\end{document}

